\href{https://github.com/bjornregnell/lth-eda016-2015/tree/master/lectures/notes}{Föreläsningsbilder
på GitHub}.
\href{http://cs.lth.se/eda016/veckoplanering}{Veckoplanering}.
\href{http://cs.lth.se/eda016/schema}{TimeEdit}.

\begin{longtable}[c]{@{}lllll@{}}
\toprule
Vecka & Föreläsning & Tema & Innehåll & Ankboken\tabularnewline
\midrule
\endhead
W01 (Lp1V1) & F1 F2 & Introduktion & Om kursen, programmeringens
grundprinciper, programmeringsparadigmer, editera-kompilera-exekvera,
datorns delar, virtuell maskin, värde, uttryck, variabel, typ,
tilldelning, utdata med System.out, indata med Scanner, alternativ, if,
else, true, false & 1.1, 1.2, 1.3, 1.4, 3.1, 3.2, 4.1, 4.2, 4.3 5.1,
5.2, 5.3, 6.1, 6.2, 7.1, 7.3\tabularnewline
W02 & F3 & Kodstruktur & loop-strukturer: while-sats, for-sats,
algoritm: min/max, Integer.MIN\_VALUE, Integer.MAX\_VALUE, Paket,
import, filstruktur, jar, dokumentation, programlayout, JDK, konstanter
vs föränderlighet, objektorientering, klasser, objekt,
referensvariabler, referenstilldelning, anropa metoder, exempel:
SimpleWindow & 2.1, 2.2, 2.3, 2.4, 2.5, 2.6, 2.8, 3.3, 4, 5.4, 7.2,
7.5-7.6, 7.8-7.9\tabularnewline
W03 & F4 & Systemutveckling & Krav-design-test, specifikationer, använda
vs implementera, exempel: Square, attribut, synlighetsregler, private,
public, konstruktor, this, implementera metoder, funktioner vs
procedurer, void, parametrar, Eclipse IDE, öppen källkod, Stack
overflow, Github & Kapitel 2.7-2.10, 3.4-3.12,\tabularnewline
W04 & F5 F6 & Aritmetik, Logik, Datastrukturer & exempel: Point &
Kapitel 6.3-6.9\tabularnewline
W05 & F7 F8 & Klasser, Strängar, Slumptal & exempel: Person, Die,
switch, char, String, do-while & Kapitel 11.1-11.3, 6.10-6.11, 7.4,
7.7\tabularnewline
W06 & F9 F10 & Vektorer, Registrering & & Kapitel 8\tabularnewline
W07 & F11 F12 & Likhet, Synlighet, StringBuilder & & Kapitel
?-?\tabularnewline
W09 (Lp2V1) & F13 & Matriser & & Kapitel 8.6-8.7\tabularnewline
W10 (Lp2V2) & F14 & Listor & ArrayList, typklasser, autoboxning &
Kapitel 12\tabularnewline
W11 (Lp2V3) & F15 F16 & Arv & & Kapitel 9\tabularnewline
W12 (Lp2V4) & F17 F18 & Algoritmer & linjärsökning, binärsökning,
urvalssortering, bubbelsortering, insättningssortering, komplexitet &
Kapitel 7.7, 8\tabularnewline
W13 (Lp2V5) & F19 & Designexempel & att skriva stora program, om
fördjupningskursen & Kapitel 10, 13, (14-16)\tabularnewline
W14 (Lp2V6) & F20 & Extentor, Repetition & &\tabularnewline
W15 (Lp2V7) & F21 & Utblick, Om tentan & framtidens systemutveckling,
kommande kurser, ämnen på begäran &\tabularnewline
\bottomrule
\end{longtable}
