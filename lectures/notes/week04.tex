\documentclass{lecturenotes}

\renewcommand{\vecka}{4}
\newcommand{\tema}{Aritmetik, Logik \& Datastrukturer}

\setbeamertemplate{footline}[frame number]
\title[Föreläsningsanteckningar EDA016, 2015]{EDA016 Programmeringsteknik för D}
\subtitle{Läsvecka \vecka: \tema}
\author{Björn Regnell}
\institute{Datavetenskap, LTH}
\date{Lp1-2, HT 2015}
 
\begin{document}

\frame{\titlepage}
\setnextsection{\vecka}
\section[Vecka \vecka: \tema]{\tema}
\frame{\tableofcontents}

%%%%%%%%%%%%%%%%%%%%%%%%%%%%%%%%%%%%%%

\subsection{Att göra denna vecka}
\frame{\frametitle{Att göra i Vecka \vecka: Förstå aritmetiska och logiska uttryck, använda klasser mha klass-specifikationer}
\begin{enumerate}
\item Läs följande kapitel i kursboken:\\ 6.1--6.4, 6.8--6.9,  7.1--7.7, 7.10--7.11, 7.13\\ 
Begrepp:heltalsdivition med rest,  typkonvertering, De Morgans lagar, oföränderlighet
\item Gör övning 4: Aritmetik, Logik
\item Träffas i samarbetsgrupper och hjälp varandra förstå 
\item Gör Lab 3: använda färdigskrivna klasser, kvadrat
\end{enumerate}
}

\begin{Slide}{Implicit och explicit konvertering mellan numeriska värden}
\end{Slide}


\begin{Slide}{Minsta och största värden}
\end{Slide}

\begin{Slide}{Några aritmetriska uttryck}
\end{Slide}

\begin{Slide}{Var n-te gång, jämt delbart med n}
\end{Slide}

\begin{Slide}{Negering av logiska uttryck med De Morgans lagar}
\end{Slide}


\begin{Slide}{Specifikation av \texttt{SimpleWindow}}
\end{Slide}

\begin{Slide}{Specifikation av klassen \texttt{Square}}
\begin{lstlisting}[basicstyle=\ttfamily\tiny\selectfont,
backgroundcolor=,rulecolor=\color{black}, title={\texttt{Square}}, frameround=tttt,language=]
/** Skapar en kvadrat med övre vänstra hörnet i x,y och med sidlängden side  */
Square(int x, int y, int side);

/** Ritar kvadraten i fönstret w */
void draw(SimpleWindow w);

/** Flyttar kvadraten avståndet dx i x-led, dy i y-led */
void move(int dx, int dy);

/** Tar reda på x-koordinaten för kvadratens läge */
int getX();

/** Tar reda på y-koordinaten för kvadratens läge */
int getY();

/** Tar reda på kvadratens area */
int getArea();
\end{lstlisting}
\end{Slide}

\begin{Slide}{Standardvärden för attribut}
\end{Slide}

\Subsection{Oföränderlighet (immutability)}
\begin{Slide}{Förhindra att variabler \href{https://docs.oracle.com/javase/tutorial/essential/concurrency/immutable.html}{ändras} med \texttt{\textbf{final}}}
Attributet \texttt{latinsktNamn} nedan är en \Emph{konstant}.\\ Kompilatorn hjälper oss att kolla så att vi inte råkar ändra på det vi har deklarerat som \Key{final}.
\lstinputlisting[language=Java, basicstyle=\ttfamily\tiny\selectfont, numberstyle=, numbers=left,]{../examples/terminal/final/Constant.java}
\end{Slide}

\begin{Slide}{Oföränderligt objekt}
\lstinputlisting[language=Java, basicstyle=\ttfamily\tiny\selectfont, numberstyle=, numbers=left,]{../examples/terminal/final/ImmutableObject.java}
\end{Slide}

\end{document}