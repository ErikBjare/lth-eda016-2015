\documentclass{lecturenotes}

\renewcommand{\vecka}{4}
\newcommand{\tema}{Systemutveckling}

\setbeamertemplate{footline}[frame number]
\title[Föreläsningsanteckningar EDA016, 2015]{EDA016 Programmeringsteknik för D}
\subtitle{Läsvecka \vecka: \tema}
\author{Björn Regnell}
\institute{Datavetenskap, LTH}
\date{Lp1-2, HT 2015}
 
\begin{document}

\frame{\titlepage}
\setnextsection{\vecka}
\section[Vecka \vecka: \tema]{\tema}
\frame{\tableofcontents}

%%%%%%%%%%%%%%%%%%%%%%%%%%%%%%%%%%%%%%

\subsection{Att göra denna vecka}
\frame{\frametitle{Att göra i Vecka \vecka: Förstå hur systemutveckling går till med klasser och objekt i en integrerad utvecklingsmiljö}
\begin{enumerate}
\item Läs följande kapitel i kursboken:\\ 2.7--2.10, 3.3--3.12\\ 
 Begrepp: oföränderlighet
\item Gör övning 4: ???
\item Träffas i samarbetsgrupper och hjälp varandra förstå 
\item Gör Lab 3: ???
\end{enumerate}
}

\begin{Slide}{Standardvärden för attribut}
\end{Slide}

\Subsection{Oföränderlighet (immutability)}
\begin{Slide}{Förhindra att variabler \href{https://docs.oracle.com/javase/tutorial/essential/concurrency/immutable.html}{ändras} med \texttt{\textbf{final}}}
Attributet \texttt{latinsktNamn} nedan är en \Emph{konstant}.\\ Kompilatorn hjälper oss att kolla så att vi inte råkar ändra på det vi har deklarerat som \Key{final}.
\lstinputlisting[language=Java, basicstyle=\ttfamily\tiny\selectfont, numberstyle=, numbers=left,]{../examples/terminal/final/Constant.java}
\end{Slide}

\begin{Slide}{Oföränderligt objekt}
\lstinputlisting[language=Java, basicstyle=\ttfamily\tiny\selectfont, numberstyle=, numbers=left,]{../examples/terminal/final/ImmutableObject.java}
\end{Slide}

\end{document}