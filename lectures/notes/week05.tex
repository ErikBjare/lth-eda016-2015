\documentclass{lecturenotes}

\renewcommand{\vecka}{5}
\newcommand{\tema}{Tecken, strängar och slumptal}

\setbeamertemplate{footline}[frame number]
\title[Föreläsningsanteckningar EDA016, 2015]{EDA016 Programmeringsteknik för D}
\subtitle{Läsvecka \vecka: \tema}
\author{Björn Regnell}
\institute{Datavetenskap, LTH}
\date{Lp1-2, HT 2015}
 
\begin{document}

\frame{\titlepage}
\setnextsection{\vecka}
\section[Vecka \vecka: \tema]{\tema}
\frame{\tableofcontents}

%%%%%%%%%%%%%%%%%%%%%%%%%%%%%%%%%%%%%%

\subsection{Att göra denna vecka}
\frame{\frametitle{Att göra i Vecka \vecka: Förstå aritmetiska och logiska uttryck, använda klasser mha klass-specifikationer}
\begin{enumerate}
\item Läs följande kapitel i kursboken: 11, \\  
Begrepp: sträng, toString, StringBuilder, slumptalsgenerator, Random
\item Gör övning 5: 
\item Gör gammal kontrollskrivning \& rätta i samarbetsgrupper 
\item Gör Lab 4: 
\end{enumerate}
}


\subsection{Tecken}
\begin{Slide}{Tecken}
\lstinputlisting[language=Java, basicstyle=\ttfamily\tiny\selectfont, numberstyle=, numbers=left,]{../examples/eclipse-ws/lecture-examples/src/week05/ShowCharacters.java}
\end{Slide}

\begin{Slide}{Specialtecken}
Some \href{https://docs.oracle.com/javase/tutorial/java/data/characters.html}{Java Escape Sequences}:
\begin{itemize}
\item Ny rad: \verb+'\n'+
\item Tab: \verb+'\t'+
\item Citationstecken: \verb+'\t'+
\item Tab: \verb+'\t'+
\item Godtyckligt unicode-tecken:  \verb+'\u03BB'+ 
\end{itemize}
Se \href{}{exempel}.
\end{Slide}
\begin{Slide}{print vs println}
\end{Slide}


\subsection{Strängar}
\subsection{Slumptal}
\subsection{Switch}
\subsection{do-while}


\end{document}