\documentclass{lecturenotes}

\renewcommand{\vecka}{2}

\title[Föreläsningsanteckningar EDA016, 2015]{EDA016 Programmeringsteknik för D}
\subtitle{Läsvecka \vecka: Kodstruktur}
\author{Björn Regnell}
\institute{Datavetenskap, LTH}
\date{Lp1-2, HT 2015}
 
\begin{document}

\frame{\titlepage}
\setnextsection{2}
\section[Vecka \vecka: Kodstruktur]{Kodstruktur}
\frame{\tableofcontents}

%%%%%%%%%%%%%%%%%%%%%%%%%%%%%%%%%%%%%%
\subsection{Varför behövs kodstruktur?}
\begin{Slide}{Varför kodstruktur?}
\begin{itemize}
\item Nu blev denna programdel för stor och behöver delas upp...
\end{itemize}
\end{Slide}

\Subsection{Objekt}

\begin{Slide}{Objekt och referensvariabler}
% http://tex.stackexchange.com/questions/45404/asymmetric-cloud-shape-in-tikz
% http://tex.stackexchange.com/questions/44940/cloud-with-lines-as-filling-in-tikz
\begin{tikzpicture}[font=\large\sffamily]
\node[cloud, cloud puffs=15.7, cloud ignores aspect, %minimum width=5cm, minimum height=2cm,
 align=center, draw] (cloud) at (0cm, 0cm) {objekt};
\end{tikzpicture}
\end{Slide}

\end{document}