\documentclass{lecturenotes}

\renewcommand{\vecka}{3}
\newcommand{\tema}{Systemutveckling}

\setbeamertemplate{footline}[frame number]
\title[Föreläsningsanteckningar EDA016, 2015]{EDA016 Programmeringsteknik för D}
\subtitle{Läsvecka \vecka: \tema}
\author{Björn Regnell}
\institute{Datavetenskap, LTH}
\date{Lp1-2, HT 2015}
 
\begin{document}

\frame{\titlepage}
\setnextsection{2}
\section[Vecka \vecka: \tema]{\tema}
\frame{\tableofcontents}

%%%%%%%%%%%%%%%%%%%%%%%%%%%%%%%%%%%%%%

\subsection{Att göra denna vecka}
\frame{\frametitle{Att göra i Vecka \vecka: Förstå hur systemutveckling går till i en integrerad utvecklingsmiljö}
\begin{enumerate}
\item Läs följande kapitel i kursboken:\\ 2, 6.3, 7.1, 7.2, 7.3, 5.1, 5.2, 5.3\\ 
 Begrepp: klass, objekt, specifikation, referensvariabel, instans, IDE, arbetsområde, brytpunkt, avlusare,   
\item Gör övning 3: beräkningar, klasser och objekt
\item Träffas i samarbetsgrupper och hjälp varandra förstå 
\item Gör Lab 2: Eclipse
\end{enumerate}
}

\Subsection{Klasser och objekt}

\begin{Slide}{Objekt och referensvariabler}
\lstinputlisting[language=Java,numbers=left]{../examples/terminal/refvars/ReferenceVariables.java}
\end{Slide}

\begin{Slide}{Objekt och referensvariabler}
\begin{lstlisting}[numbers=left, firstnumber=7]
        Gurka g1 = new Gurka();
        Gurka g2 = new Gurka();
\end{lstlisting}
Efter rad 8 ser det ut såhär i minnet:
\\ \vspace{1em}
\begin{tikzpicture}[font=\large\sffamily]
\matrix [matrix of nodes, row sep=0, column 2/.style={nodes={rectangle,draw,minimum width=3em}}] (mat)
{
\texttt{g1}   &  \makebox(16,12){ }\\
\texttt{g2}   &  \makebox(16,12){ }\\
};
\node[cloud, cloud puffs=15.7, cloud ignores aspect, %minimum width=5cm, minimum height=2cm,
 align=center, draw] (g1) at (4cm, 1cm) {Gurka-\\objekt};
\filldraw[black] (0.4cm,0.4cm) circle (3pt) node[] (g1ref) {};
 \draw [arrow] (g1ref) -- (g1);
 \node[cloud, cloud puffs=15.7, cloud ignores aspect, %minimum width=5cm, minimum height=2cm,
 align=center, draw] (g2) at (5cm, -2cm) {Gurka-\\objekt};
 \filldraw[black] (0.4cm,-0.4cm) circle (3pt) node[] (g2ref) {};
 \draw [arrow] (g2ref) -- (g2);
\end{tikzpicture}
\end{Slide}


\begin{Slide}{Objekt och referensvariabler}
\begin{lstlisting}[numbers=left, firstnumber=7]
        Gurka g1 = new Gurka();
        Gurka g2 = new Gurka();
\end{lstlisting}
\scriptsize En mer detaljerad bild av minnet efter rad 8:
\\ \vspace{1em}
\tikzstyle{mybox} = [draw=red, fill=blue!20, very thick,
    rectangle, rounded corners, inner sep=10pt, inner ysep=20pt]
\begin{tikzpicture}[
      scale = 0.8, every node/.style={transform shape},
	font=\large\sffamily, 
	varname/.style={node distance=0.2cm},
	varbox/.style={draw, node distance=0.2cm},
	objcloud/.style={cloud, cloud puffs=15.7, cloud ignores aspect, align=center, draw},
]

\node [varname] (g1var) {\texttt{g1}};
\node [varbox, right = of g1var] (g1ref) {\phantom{abc}};
\filldraw[black] (g1ref) circle (3pt) node[] (g1dot) {};
\node [objcloud, right = of g1ref, yshift=1.0cm, scale =0.8] (g1obj) {
	\texttt{\textbf{Gurka}} \\~\\ \texttt{vikt} \framebox{100}
};
\draw [arrow] (g1dot) -- (g1obj);

\node [varname, below = of g1var] (g2var) {\texttt{g2}};
\node [varbox, right = of g2var] (g2ref) {\phantom{abc}};
\filldraw[black] (g2ref) circle (3pt) node[] (g2dot) {};
\node [objcloud, right = of g2ref, yshift=-1.0cm, scale =0.8] (g2obj) {
	\texttt{\textbf{Gurka}} \\~\\ \texttt{vikt} \framebox{100}
};
\draw [arrow] (g2dot) -- (g2obj);

\end{tikzpicture}

\pause\scriptsize Referensvariablerna \texttt{g1} och \texttt{g2} pekar på \textit{olika} objekt, sålunda är uttrycket \texttt{g1 == g2} \Key{false}, även om objektens \textit{innehåll} är lika och \texttt{g1.vikt == g2.vikt}  är \Key{true}.
\end{Slide}


\begin{Slide}{Punktnotation för att komma åt klassmedlemmar}
\begin{lstlisting}[numbers=left, firstnumber=9]
        g2.vikt = 200;
\end{lstlisting}
Efter rad 9 ser det ut såhär i minnet:
\\ \vspace{1em}
\tikzstyle{mybox} = [draw=red, fill=blue!20, very thick,
    rectangle, rounded corners, inner sep=10pt, inner ysep=20pt]
\begin{tikzpicture}[
      scale = 0.8, every node/.style={transform shape},
	font=\large\sffamily, 
	varname/.style={node distance=0.2cm},
	varbox/.style={draw, node distance=0.2cm},
	objcloud/.style={cloud, cloud puffs=15.7, cloud ignores aspect, align=center, draw},
]

\node [varname] (g1var) {\texttt{g1}};
\node [varbox, right = of g1var] (g1ref) {\phantom{abc}};
\filldraw[black] (g1ref) circle (3pt) node[] (g1dot) {};
\node [objcloud, right = of g1ref, yshift=1.0cm, scale =0.8] (g1obj) {
	\texttt{\textbf{Gurka}} \\~\\ \texttt{vikt} \framebox{100}
};
\draw [arrow] (g1dot) -- (g1obj);

\node [varname, below = of g1var] (g2var) {\texttt{g2}};
\node [varbox, right = of g2var] (g2ref) {\phantom{abc}};
\filldraw[black] (g2ref) circle (3pt) node[] (g2dot) {};
\node [objcloud, right = of g2ref, yshift=-1.0cm, scale =0.8] (g2obj) {
	\texttt{\textbf{Gurka}} \\~\\ \texttt{vikt} \framebox{200}
};
\draw [arrow] (g2dot) -- (g2obj);

\end{tikzpicture}
\end{Slide}

\Subsection{Metoder och parametrar}

\begin{Slide}{Deklarera och anropa metoder}
\lstinputlisting[language=Java,basicstyle=\ttfamily\tiny\selectfont, numberstyle=, numbers=left,]{../examples/terminal/methods/MethodExamples.java}
\end{Slide}

\begin{Slide}{Parametrar}
\lstinputlisting[language=Java,basicstyle=\ttfamily\tiny\selectfont, numberstyle=, numbers=left,]{../examples/terminal/methods/MethodWithParameter.java}
\end{Slide}

\begin{Slide}{Lokala namn och överskuggning}
\scriptsize Vad händer här? OBS! Tre \textit{olika} variabler, men med samma namn...
\lstinputlisting[language=Java,basicstyle=\ttfamily\tiny\selectfont, numberstyle=, numbers=left,]{../examples/terminal/localvars/LocalVar.java}
\end{Slide}

\Subsection{Synlighet}
\begin{Slide}{Public och Private}
\end{Slide}

\Subsection{Konstruktorer}
\begin{Slide}{This}
\end{Slide}

\begin{Slide}{Aktiveringspost}
\end{Slide}

\Subsection{Oföränderlighet (immutability)}
\begin{Slide}{Förhindra att variabler \href{https://docs.oracle.com/javase/tutorial/essential/concurrency/immutable.html}{ändras} med \texttt{\textbf{final}}}
Attributet \texttt{latinsktNamn} nedan är en \Emph{konstant}.\\ Kompilatorn hjälper oss att kolla så att vi inte råkar ändra på det vi har deklarerat som \Key{final}.
\lstinputlisting[language=Java, basicstyle=\ttfamily\tiny\selectfont, numberstyle=, numbers=left,]{../examples/terminal/final/Constant.java}
\end{Slide}

\Subsection{Specifikation versus implementation}
\begin{Slide}{Krav -- Design -- Implementation -- Test}
\end{Slide}

\Subsection{Integrerad utvecklingsmiljö}
\begin{Slide}{Eclipse}
\end{Slide}

\end{document}
