\begin{longtable}[c]{@{}ll@{}}
\toprule\addlinespace
\textbf{EDA016} & D1, 7,5 högskolepoäng, Läsperiod 1 \& 2
\\\addlinespace
\midrule\endhead
\emph{Kursansvarig} & Björn Regnell, rum E:2413,
\href{mailto:bjorn.regnell@cs.lth.se}{bjorn.regnell@cs.lth.se}, 046--222
90 09
\\\addlinespace
\emph{Expedition} & Lena Ohlsson, rum E:2179,
\href{mailto:lena.ohlsson@cs.lth.se}{lena.ohlsson@cs.lth.se}, 046--222
80 40
\\\addlinespace
& Expeditionstid 9.30--11.30 och 12.45--13.30
\\\addlinespace
\emph{Hemsida} & \url{http://cs.lth.se/eda016}
\\\addlinespace
\emph{Kurslitteratur} & Per Holm: \emph{Objektorienterad programmering
och Java}, kap 1-13
\\\addlinespace
& ISBN 978-91-44-04830-7, tredje uppl., Studentlitteratur 2007.
\\\addlinespace
& \emph{Kompendium} med övningar, laborationer, inlämningsuppgifter
\\\addlinespace
& Boken och kompendiet säljs av \url{http://www.kfsab.se/}
\\\addlinespace
\bottomrule
\end{longtable}

\subsection{Undervisning}\label{undervisning}

\begin{itemize}
\item
  \emph{Föreläsningar}. Föreläsningarna ger en översikt av
  kursinnehållet och åskådliggör teorin med praktiska
  programmeringsexempel. Föreläsningarna ger även utrymme för diskussion
  och frågor.
\item
  \emph{Resurstider}. I kursens schema finns särskilda s.k. resurstider
  där du kan få hjälp med övningar, laborationer och
  inlämningsuppgifter. Utnyttja dessa tillfällen!
\item
  \emph{Övningar}. I kursen ingår 11 övningar som du arbetar med
  självständigt. Du kan få hjälp\\med övningarna under resurstiderna.
  Övningarna hjälper dig att förbereda dig inför laborationerna och den
  skriftliga tentamen. Se anvisningar i kompendiet.
\item
  \emph{Laborationer}. I Kursen ingår 11 obligatoriska laborationer,
  varav 9 görs individuellt och 2 görs i samarbetsgrupper. Se
  anvisningar i kompendiet.
\item
  \emph{Inlämningsuppgift}. Du ska självständigt arbeta med ett större
  program och redovisa detta för en handledare. Se anvisningar i
  kompendiet.
\end{itemize}

\subsection{Samarbetsgrupper}\label{samarbetsgrupper}

Kursdeltagarna indelas i \emph{samarbetsgrupper} av kursansvarig baserat
på förkunskapsenkät, där 4-5 studenter med olika förkunskapsnivåer
sammanförs. Målet med samarbetsgrupperna är att deltagarna gemensamt ska
dela med sig av och träna på förklaringar av teori, begrepp och
programmeringspraktik. Kontrollskrivningen kan ge samarbetsbonus (se
nedan) och 2 av laborationerna görs i grupp. Ni ska hjälpa varandra att
förstå, men \emph{inte} lösa uppgifterna åt varandra.

\subsection{Examination}\label{examination}

\begin{itemize}
\item
  \emph{Obligatoriska kursmoment (4,5 hp)}. Laborationer,
  kontrollskrivning och inlämningsuppgift.

  \begin{itemize}
  \item
    Laborationer godkänns av handledare på schemalagd tid. Se
    instruktioner i kompendium.
  \item
    Kontrollskrivningen är diagnostisk och visar ditt kunskapsläge efter
    halva kursen. Kontrollskrivningen görs individuellt och rättas
    därefter av studiekamrater vid skrivningstillfället.
    Kontrollskrivningens kan ge \emph{samarbetsbonus} som adderas till
    det skriftliga tentamensresultatet med medelvärdet av
    gruppmedlemmarnas individuella kontrollskrivningspoäng, max 3
    bonuspoäng.
  \item
    Inlämningsuppgift görs individuellt och godkänns av handledare på
    schemalagd tid. Se instruktioner i kompendium.
  \end{itemize}
\item
  \emph{Tentamen (3 hp)}. Tentamen är skriftlig. Tillåtet hjälpmedel:
  Java \href{http://cs.lth.se/eda016/javaref}{snabbreferens}. För att få
  tentera krävs att laborationerna och inlämningsuppgifterna är
  godkända.

  \begin{itemize}
  \itemsep1pt\parskip0pt\parsep0pt
  \item
    Ordinarie tentamen: Onsdagen den 13 Januari, 2016, sal: MA10 i
    Matteannexet.
  \end{itemize}
\end{itemize}
