För tider, salar och grupper se schema i
\href{http://cs.lth.se/eda016/schema}{TimeEdit}. Denna veckoplan på
\href{https://github.com/bjornregnell/lth-eda016-2015/blob/master/weekplan/weekplan.md}{GitHub}.

\subsubsection{Läsperiod 1}\label{lasperiod-1}

\begin{longtable}[c]{@{}llllll@{}}
\toprule\addlinespace
Vecka & Datum & Tema & Föreläsning & Resurstid & Laboration
\\\addlinespace
\midrule\endhead
Lp1V1 & 31/8-6/9 & Introduktion & F1 F2 & Ö1 hello & Lab1 Quiz
\\\addlinespace
Lp1V2 & & Kodstruktur & F3 - & Ö2 paket, kodfiler & --
\\\addlinespace
Lp1V3 & & Systemutveckling & F4 -- & Ö3 beräkn., klasser, objekt & Lab2
Eclipse
\\\addlinespace
Lp1V4 & & Aritmetik & F5 F6 & Ö4 aritmetik, logik & Lab3 Anv. Square
\\\addlinespace
Lp1V5 & & Klasser & F7 F8 & Ö5 klasser, slumptal & Lab4 Impl. Square
\\\addlinespace
Lp1V6 & & Vektorer & F9 F10 & Ö6 vektor, registrering & Lab5 Gissa Tal
\\\addlinespace
Lp1V7 & & Likhet, synlighet & F11 F12 & Ö7 registrering & Lab6 Turtle
\\\addlinespace
& Tis 27/10 & KS & -- & -- & --
\\\addlinespace
\bottomrule
\end{longtable}

KS = Kontrollskrivning; obligatorisk, diagnostisk, kamraträttad, kan ge
samarbetsbonus.

\subsubsection{Läsperiod 2}\label{lasperiod-2}

\begin{longtable}[c]{@{}llllll@{}}
\toprule\addlinespace
Vecka & Datum & Tema & Föreläsning & Resurstid & Laboration
\\\addlinespace
\midrule\endhead
Lp2V1 & 2/11-8/11 & Matriser & F13 -- & Ö8 matriser, StringBuilder &
Lab7 Maze
\\\addlinespace
Lp2V2 & & Listor & F14 -- & Ö9 ArrayList & Lab8 Vektor
\\\addlinespace
Lp2V3 & & Arv & F15 F16 & Ö10 arv & Lab9 grupplab TurtleRace
\\\addlinespace
Lp2V4 & & Algoritmer & F17 F18 & Ö11 sortering, objekt & Lab10 Life
\\\addlinespace
Lp2V5 & & Designexempel & F19 -- & extraövningar, extentor & Lab11
grupplab Imagefilter
\\\addlinespace
Lp2V6 & & Extentor, Repetition & F20 -- & extraövningar, extentor &
Inlämningsuppgift
\\\addlinespace
Lp2V7 & & Utblick, Om tentan & F21 -- & uppsamling & --
\\\addlinespace
& Ons 13/1 & \emph{Tenta} & -- & -- & --
\\\addlinespace
\bottomrule
\end{longtable}

Tenta = Skriftlig tentamen utan hjälpmedel, förutom
\href{http://cs.lth.se/eda016/javaref}{snabbreferens}.
